% Options for packages loaded elsewhere
\PassOptionsToPackage{unicode}{hyperref}
\PassOptionsToPackage{hyphens}{url}
%
\documentclass[
]{article}
\usepackage{lmodern}
\usepackage{amssymb,amsmath}
\usepackage{ifxetex,ifluatex}
\ifnum 0\ifxetex 1\fi\ifluatex 1\fi=0 % if pdftex
  \usepackage[T1]{fontenc}
  \usepackage[utf8]{inputenc}
  \usepackage{textcomp} % provide euro and other symbols
\else % if luatex or xetex
  \usepackage{unicode-math}
  \defaultfontfeatures{Scale=MatchLowercase}
  \defaultfontfeatures[\rmfamily]{Ligatures=TeX,Scale=1}
\fi
% Use upquote if available, for straight quotes in verbatim environments
\IfFileExists{upquote.sty}{\usepackage{upquote}}{}
\IfFileExists{microtype.sty}{% use microtype if available
  \usepackage[]{microtype}
  \UseMicrotypeSet[protrusion]{basicmath} % disable protrusion for tt fonts
}{}
\makeatletter
\@ifundefined{KOMAClassName}{% if non-KOMA class
  \IfFileExists{parskip.sty}{%
    \usepackage{parskip}
  }{% else
    \setlength{\parindent}{0pt}
    \setlength{\parskip}{6pt plus 2pt minus 1pt}}
}{% if KOMA class
  \KOMAoptions{parskip=half}}
\makeatother
\usepackage{xcolor}
\IfFileExists{xurl.sty}{\usepackage{xurl}}{} % add URL line breaks if available
\IfFileExists{bookmark.sty}{\usepackage{bookmark}}{\usepackage{hyperref}}
\hypersetup{
  pdftitle={Oblig2-TMA4300},
  pdfauthor={Martine Middelthon},
  hidelinks,
  pdfcreator={LaTeX via pandoc}}
\urlstyle{same} % disable monospaced font for URLs
\usepackage[margin=1in]{geometry}
\usepackage{color}
\usepackage{fancyvrb}
\newcommand{\VerbBar}{|}
\newcommand{\VERB}{\Verb[commandchars=\\\{\}]}
\DefineVerbatimEnvironment{Highlighting}{Verbatim}{commandchars=\\\{\}}
% Add ',fontsize=\small' for more characters per line
\usepackage{framed}
\definecolor{shadecolor}{RGB}{248,248,248}
\newenvironment{Shaded}{\begin{snugshade}}{\end{snugshade}}
\newcommand{\AlertTok}[1]{\textcolor[rgb]{0.94,0.16,0.16}{#1}}
\newcommand{\AnnotationTok}[1]{\textcolor[rgb]{0.56,0.35,0.01}{\textbf{\textit{#1}}}}
\newcommand{\AttributeTok}[1]{\textcolor[rgb]{0.77,0.63,0.00}{#1}}
\newcommand{\BaseNTok}[1]{\textcolor[rgb]{0.00,0.00,0.81}{#1}}
\newcommand{\BuiltInTok}[1]{#1}
\newcommand{\CharTok}[1]{\textcolor[rgb]{0.31,0.60,0.02}{#1}}
\newcommand{\CommentTok}[1]{\textcolor[rgb]{0.56,0.35,0.01}{\textit{#1}}}
\newcommand{\CommentVarTok}[1]{\textcolor[rgb]{0.56,0.35,0.01}{\textbf{\textit{#1}}}}
\newcommand{\ConstantTok}[1]{\textcolor[rgb]{0.00,0.00,0.00}{#1}}
\newcommand{\ControlFlowTok}[1]{\textcolor[rgb]{0.13,0.29,0.53}{\textbf{#1}}}
\newcommand{\DataTypeTok}[1]{\textcolor[rgb]{0.13,0.29,0.53}{#1}}
\newcommand{\DecValTok}[1]{\textcolor[rgb]{0.00,0.00,0.81}{#1}}
\newcommand{\DocumentationTok}[1]{\textcolor[rgb]{0.56,0.35,0.01}{\textbf{\textit{#1}}}}
\newcommand{\ErrorTok}[1]{\textcolor[rgb]{0.64,0.00,0.00}{\textbf{#1}}}
\newcommand{\ExtensionTok}[1]{#1}
\newcommand{\FloatTok}[1]{\textcolor[rgb]{0.00,0.00,0.81}{#1}}
\newcommand{\FunctionTok}[1]{\textcolor[rgb]{0.00,0.00,0.00}{#1}}
\newcommand{\ImportTok}[1]{#1}
\newcommand{\InformationTok}[1]{\textcolor[rgb]{0.56,0.35,0.01}{\textbf{\textit{#1}}}}
\newcommand{\KeywordTok}[1]{\textcolor[rgb]{0.13,0.29,0.53}{\textbf{#1}}}
\newcommand{\NormalTok}[1]{#1}
\newcommand{\OperatorTok}[1]{\textcolor[rgb]{0.81,0.36,0.00}{\textbf{#1}}}
\newcommand{\OtherTok}[1]{\textcolor[rgb]{0.56,0.35,0.01}{#1}}
\newcommand{\PreprocessorTok}[1]{\textcolor[rgb]{0.56,0.35,0.01}{\textit{#1}}}
\newcommand{\RegionMarkerTok}[1]{#1}
\newcommand{\SpecialCharTok}[1]{\textcolor[rgb]{0.00,0.00,0.00}{#1}}
\newcommand{\SpecialStringTok}[1]{\textcolor[rgb]{0.31,0.60,0.02}{#1}}
\newcommand{\StringTok}[1]{\textcolor[rgb]{0.31,0.60,0.02}{#1}}
\newcommand{\VariableTok}[1]{\textcolor[rgb]{0.00,0.00,0.00}{#1}}
\newcommand{\VerbatimStringTok}[1]{\textcolor[rgb]{0.31,0.60,0.02}{#1}}
\newcommand{\WarningTok}[1]{\textcolor[rgb]{0.56,0.35,0.01}{\textbf{\textit{#1}}}}
\usepackage{graphicx,grffile}
\makeatletter
\def\maxwidth{\ifdim\Gin@nat@width>\linewidth\linewidth\else\Gin@nat@width\fi}
\def\maxheight{\ifdim\Gin@nat@height>\textheight\textheight\else\Gin@nat@height\fi}
\makeatother
% Scale images if necessary, so that they will not overflow the page
% margins by default, and it is still possible to overwrite the defaults
% using explicit options in \includegraphics[width, height, ...]{}
\setkeys{Gin}{width=\maxwidth,height=\maxheight,keepaspectratio}
% Set default figure placement to htbp
\makeatletter
\def\fps@figure{htbp}
\makeatother
\setlength{\emergencystretch}{3em} % prevent overfull lines
\providecommand{\tightlist}{%
  \setlength{\itemsep}{0pt}\setlength{\parskip}{0pt}}
\setcounter{secnumdepth}{-\maxdimen} % remove section numbering

\title{Oblig2-TMA4300}
\author{Martine Middelthon}
\date{27 2 2020}

\begin{document}
\maketitle

\hypertarget{problem-b}{%
\section{Problem B}\label{problem-b}}

\begin{Shaded}
\begin{Highlighting}[]
\CommentTok{# Read and plot data}
\NormalTok{gaussiandata =}\StringTok{ }\KeywordTok{read.delim}\NormalTok{(}\StringTok{"gaussiandata.txt"}\NormalTok{)}
\NormalTok{y =}\StringTok{ }\NormalTok{gaussiandata[,}\DecValTok{1}\NormalTok{]}
\NormalTok{t =}\StringTok{ }\KeywordTok{seq}\NormalTok{(}\DataTypeTok{from=}\DecValTok{1}\NormalTok{,}\DataTypeTok{to=}\KeywordTok{length}\NormalTok{(y),}\DataTypeTok{by=}\DecValTok{1}\NormalTok{)}
\KeywordTok{plot}\NormalTok{(t,y)}
\end{Highlighting}
\end{Shaded}

\includegraphics{Oblig2-problemB_files/figure-latex/unnamed-chunk-1-1.pdf}

\hypertarget{section}{%
\subsection{1.}\label{section}}

We consider the problem of smoothing the time series that is plotted
above. We assume that given the vector of linear predictors
\(\boldsymbol\eta=(\eta_1,\ldots,\eta_T)\), where in this case \(T=20\),
the observations \(y_t\) are independent and distributed according to \[
  y_t \mid \eta_t \sim \mathcal{N}(\eta_t,1)\quad,
\] for \(t=1,\ldots,T\). The linear predictor for time \(t\) is
\(\eta_t = f_t\), where \(f_t\) is the smooth effect for time \(t\). For
the prior distribution of \(\mathbf{f}=(f_1,\ldots,f_T)\) we have a
second order random walk model, that is, \[
  \pi(\mathbf{f}\mid\theta) \propto \theta^{(T-2)/2} \text{exp}\Big\{-\frac{\theta}{2} \sum_{t=3}^T (f_t-2f_{t-1}+f_{t-2}^2) \Big\}=\mathcal{N}(\mathbf{0},\mathbf{Q}(\theta)^{-1}) \quad,
\] where \(\mathbf{Q}\) is the precision matrix and \(\theta\) is the
precision parameter that controls the smoothness of \(\mathbf{f}\). We
assume that the \(Gamma(1,1)\)-distribution is the prior for \(\theta\).

The model described here can be written as the hierarchichal model: \[
  \begin{aligned}
    \mathbf{y}\mid\mathbf{f} &\sim \prod_{t=1}^T P(y_t\mid \eta_t) \\
    \mathbf{f}\mid\theta &\sim \pi(\mathbf{f}\mid\theta) = \mathcal{N}(\mathbf{0},\mathbf{Q}(\theta)^{-1}) \\
    \theta &\sim Gamma(1,1)
  \end{aligned}
\] Here, the first line is the likelihood of the response
\(\mathbf{y}=(y_1,\ldots,y_T)\), the second line gives the prior
distribution of the latent field, and the third line gives the prior
distribution of the hyperparameter \(\theta\). Since our model has this
particular structure, it is a latent Gaussian model. INLA can be used to
estimate the parameters because we have a latent gaussian model where
each data point \(y_t\) depends only on the one element \(f_t\) in the
latent field, the dimension of the hyperparameter is one and the
precision matrix \(\mathbf{Q}(\theta)\) of the latent field is sparse.

\hypertarget{section-1}{%
\subsection{2.}\label{section-1}}

Here, we implement a block Gibbs sampling algorithm for
\(f(\boldsymbol\eta,\theta\mid \mathbf{y})\), where we propose a new
value for \(\theta\) from the full conditional
\(\pi(\theta\mid\boldsymbol\eta,\mathbf{y})\) and a new value for
\(\boldsymbol\eta\) from the full conditional
\(\pi(\boldsymbol\eta\mid\theta,\mathbf{y})\). Thus, we need to find
these distributions. We start with the posterior \[
  \pi(\boldsymbol\eta,\theta\mid\mathbf{y}) \propto \pi(\theta) \pi(\boldsymbol\eta\mid\theta) \prod_{t=1}^T\pi(y_t\mid\eta_t,\theta) \propto \frac{\theta^{(T-2)/2}}{(2\pi)^{T/2}} \exp\bigg\{-\theta -\frac{\theta}{2}\sum_{t=3}^T(\eta_t-2\eta_{t-1}+\eta_{t-2})^2 -\frac{1}{2}\sum_{t=1}^T(y_t-\eta_t)^2 \bigg\}.
\] Then we find the full conditional for \(\theta\) to be \[
  \begin{aligned}
    \pi(\theta\mid\mathbf{y},\boldsymbol\eta) &\propto \theta^{T/2-1} \exp\bigg\{-\theta\bigg(1+\frac{1}{2} \sum_{t=3}^T(\eta_t-2\eta_{t-1}+\eta_{t-2})^2\bigg) \bigg\} \\
    &\propto \textit{Gamma}\bigg(\frac{T}{2}, 1+\frac{1}{2} \sum_{t=3}^T(\eta_t-2\eta_{t-1}+\eta_{t-2})^2 \bigg)
  \end{aligned}.
\] The full conditional for \(\boldsymbol\eta\) is \[
  \begin{aligned}
    \pi(\boldsymbol\eta\mid\theta,\mathbf{y}) &\propto \exp\bigg\{-\frac{\theta}{2} \sum_{t=3}^T(\eta_t-2\eta_{t-1}+\eta_{t-2})^2 -\frac{1}{2}\sum_{t=1}^T(y_t-\eta_t)^2 \bigg\} \\
    &= \exp\bigg\{-\frac{1}{2}\bigg(\boldsymbol\eta^T\mathbf{Q}\boldsymbol\eta+ (\mathbf{y}-\boldsymbol\eta)^T(\mathbf{y}-\boldsymbol\eta)  \bigg)\bigg\}\\
    &= \exp\bigg\{-\frac{1}{2}\boldsymbol\eta^T(\mathbf{Q}+\mathbf{I})\boldsymbol\eta+\mathbf{y}^T\boldsymbol\eta \bigg\}
  \end{aligned}.
\] Here, \(\mathbf{Q}(\theta)=\theta \mathbf{L}\mathbf{L}^T\) is the
precision matrix, where \(\mathbf{L}\) is the \(T\times (T-2)\) matrix
\[
  \mathbf{L}=
  \begin{bmatrix}
    1 & -2 & 1 & 0 & 0 & 0 & \ldots \\
    0 & 1 & -2 & 1 & 0 &0&\ldots\\
    \vdots & & \ddots &\ddots & \ddots \\
    0 & 0 & 0 & 0 & 1 & -2 & 1 \\
    0 & 0 & 0 & 0 & 0 & 1 & -2 \\
    0 & 0 & 0 & 0 & 0 & 0 & 1
  \end{bmatrix}.
\] By looking at the last line in the above expression for
\(\pi(\boldsymbol\eta\mid\theta,\mathbf{y})\), we recognize that the
canonical parametrization is
\(\mathcal{N}(\mathbf{y}, \mathbf{Q}+\mathbf{I})\), and find that
\(\pi(\boldsymbol\eta\mid\theta,\mathbf{y}) \propto \mathcal{N}((\mathbf{Q}+\mathbf{I})^{-1}\mathbf{y},(\mathbf{Q}+\mathbf{I})^{-1})\).
In the algorithm we sample the new proposals for the parameters from
these two distributions that we have found for the full conditionals. We
always use the last updated parameters.

\begin{Shaded}
\begin{Highlighting}[]
\KeywordTok{library}\NormalTok{(Matrix)}
\KeywordTok{library}\NormalTok{(mvtnorm)}
\KeywordTok{library}\NormalTok{(MASS)}

\CommentTok{# Function to make the precision matrix}
\NormalTok{make.Q =}\StringTok{ }\ControlFlowTok{function}\NormalTok{(T, theta) \{}
  \CommentTok{# Make the matrix L as described in the text}
\NormalTok{  L =}\StringTok{ }\KeywordTok{diag}\NormalTok{(T)}
\NormalTok{  d1 =}\StringTok{ }\KeywordTok{rep}\NormalTok{(}\OperatorTok{-}\DecValTok{2}\NormalTok{,T}\DecValTok{-1}\NormalTok{)}
\NormalTok{  d2 =}\StringTok{ }\KeywordTok{rep}\NormalTok{(}\DecValTok{1}\NormalTok{, T}\DecValTok{-2}\NormalTok{)}
\NormalTok{  L[}\KeywordTok{row}\NormalTok{(L)}\OperatorTok{-}\KeywordTok{col}\NormalTok{(L)}\OperatorTok{==}\DecValTok{1}\NormalTok{] =}\StringTok{ }\NormalTok{d1}
\NormalTok{  L[}\KeywordTok{row}\NormalTok{(L)}\OperatorTok{-}\KeywordTok{col}\NormalTok{(L)}\OperatorTok{==}\DecValTok{2}\NormalTok{] =}\StringTok{ }\NormalTok{d2}
\NormalTok{  L =}\StringTok{ }\NormalTok{L[,}\OperatorTok{-}\KeywordTok{c}\NormalTok{(T}\DecValTok{-1}\NormalTok{,T)]}
  \CommentTok{# Compute Q(theta)}
\NormalTok{  Q =}\StringTok{ }\NormalTok{theta }\OperatorTok{*}\StringTok{ }\NormalTok{L }\OperatorTok\StringTok{ }\KeywordTok{t}\NormalTok{(L)}
  \KeywordTok{return}\NormalTok{(Q)}
\NormalTok{\}}

\CommentTok{# Function for block Gibbs sampling}
\CommentTok{# n is the number of samples including the inital value}
\NormalTok{sample.Gibbs =}\StringTok{ }\ControlFlowTok{function}\NormalTok{(n, theta.init, f.init, y) \{}
\NormalTok{  T =}\StringTok{ }\KeywordTok{length}\NormalTok{(f.init)}
  \CommentTok{# Make vector and matrix for storing the samples}
\NormalTok{  theta.vec =}\StringTok{ }\KeywordTok{rep}\NormalTok{(}\DecValTok{0}\NormalTok{,n)}
\NormalTok{  f.matrix =}\StringTok{ }\KeywordTok{matrix}\NormalTok{(}\DecValTok{1}\OperatorTok{:}\NormalTok{T}\OperatorTok{*}\NormalTok{n, }\DataTypeTok{nrow =}\NormalTok{ T, }\DataTypeTok{ncol =}\NormalTok{ n)}
  \CommentTok{# Initialize }
\NormalTok{  theta.vec[}\DecValTok{1}\NormalTok{] =}\StringTok{ }\NormalTok{theta.init}
\NormalTok{  f.matrix[, }\DecValTok{1}\NormalTok{] =}\StringTok{ }\NormalTok{f.init}
  \CommentTok{# Iterations}
  \ControlFlowTok{for}\NormalTok{(i }\ControlFlowTok{in} \DecValTok{2}\OperatorTok{:}\NormalTok{n) \{}
    \CommentTok{# Sample theta}
\NormalTok{    summ =}\StringTok{ }\DecValTok{0}
    \ControlFlowTok{for}\NormalTok{(t }\ControlFlowTok{in} \DecValTok{3}\OperatorTok{:}\NormalTok{T) \{}
\NormalTok{      summ =}\StringTok{ }\NormalTok{summ }\OperatorTok{+}\StringTok{ }\NormalTok{(f.matrix[t, i}\DecValTok{-1}\NormalTok{] }\OperatorTok{-}\StringTok{ }\DecValTok{2}\OperatorTok{*}\NormalTok{f.matrix[t}\DecValTok{-1}\NormalTok{, i}\DecValTok{-1}\NormalTok{] }\OperatorTok{+}\StringTok{ }\NormalTok{f.matrix[t}\DecValTok{-2}\NormalTok{, i}\DecValTok{-1}\NormalTok{])}\OperatorTok{^}\DecValTok{2}
\NormalTok{    \}}
\NormalTok{    theta.vec[i] =}\StringTok{ }\KeywordTok{rgamma}\NormalTok{(}\DecValTok{1}\NormalTok{, }\DataTypeTok{shape =}\NormalTok{ T}\OperatorTok{/}\DecValTok{2}\NormalTok{, }\DataTypeTok{rate =} \DecValTok{1} \OperatorTok{+}\StringTok{ }\FloatTok{0.5}\OperatorTok{*}\NormalTok{summ)}
    \CommentTok{# Sample f}
\NormalTok{    Q =}\StringTok{ }\KeywordTok{make.Q}\NormalTok{(T, theta.vec[i])         }\CommentTok{# Use the last updated theta}
\NormalTok{    f.mean =}\StringTok{ }\KeywordTok{solve}\NormalTok{(Q}\OperatorTok{+}\KeywordTok{diag}\NormalTok{(T)) }\OperatorTok\StringTok{ }\NormalTok{y}
\NormalTok{    f.sigma =}\StringTok{ }\KeywordTok{solve}\NormalTok{(Q}\OperatorTok{+}\KeywordTok{diag}\NormalTok{(T))}
\NormalTok{    f.matrix[, i] =}\StringTok{ }\KeywordTok{rmvnorm}\NormalTok{(}\DecValTok{1}\NormalTok{, f.mean, f.sigma)}
\NormalTok{  \}}
  \KeywordTok{return}\NormalTok{(}\KeywordTok{rbind}\NormalTok{(f.matrix, theta.vec))    }\CommentTok{# Return concatenated matrix with f and theta samples}
\NormalTok{\}}

\CommentTok{# Set values}
\NormalTok{n =}\StringTok{ }\DecValTok{10000}
\NormalTok{T =}\StringTok{ }\KeywordTok{length}\NormalTok{(y)}
\NormalTok{theta.init =}\StringTok{ }\DecValTok{1}
\NormalTok{f.init =}\StringTok{ }\KeywordTok{rep}\NormalTok{(}\DecValTok{2}\NormalTok{,T)}
\CommentTok{# Sample}
\NormalTok{result =}\StringTok{ }\KeywordTok{sample.Gibbs}\NormalTok{(n, theta.init, f.init, y)}
\NormalTok{result.theta =}\StringTok{ }\NormalTok{result[}\KeywordTok{length}\NormalTok{(result[,}\DecValTok{1}\NormalTok{]), }\OperatorTok{-}\KeywordTok{c}\NormalTok{(}\DecValTok{1}\OperatorTok{:}\DecValTok{100}\NormalTok{)]   }\CommentTok{# Extracting the theta samples, excluding the first 100 values}
\NormalTok{result.f =}\StringTok{ }\NormalTok{result[}\OperatorTok{-}\KeywordTok{length}\NormalTok{(result[,}\DecValTok{1}\NormalTok{]), }\OperatorTok{-}\KeywordTok{c}\NormalTok{(}\DecValTok{1}\OperatorTok{:}\DecValTok{100}\NormalTok{)]      }\CommentTok{# Extracting the f samples, excluding the first 100 values}

\CommentTok{# Estimate for the posterior marginal for theta}
\KeywordTok{truehist}\NormalTok{(result.theta, }\DataTypeTok{xlab =} \StringTok{"Theta"}\NormalTok{, }\DataTypeTok{main =} \StringTok{"Histogram of theta samples"}\NormalTok{)}
\end{Highlighting}
\end{Shaded}

\includegraphics{Oblig2-problemB_files/figure-latex/unnamed-chunk-2-1.pdf}

\begin{Shaded}
\begin{Highlighting}[]
\CommentTok{# Vectors for storing the mean, variance and confidence bounds}
\NormalTok{f.mean =}\StringTok{ }\KeywordTok{rep}\NormalTok{(}\DecValTok{0}\NormalTok{,T)}
\NormalTok{f.var =}\StringTok{ }\KeywordTok{rep}\NormalTok{(}\DecValTok{0}\NormalTok{,T)}
\NormalTok{conf.upper =}\StringTok{ }\KeywordTok{rep}\NormalTok{(}\DecValTok{0}\NormalTok{,T)}
\NormalTok{conf.lower =}\StringTok{ }\KeywordTok{rep}\NormalTok{(}\DecValTok{0}\NormalTok{,T)}
\CommentTok{# Calculate the mean and variance}
\ControlFlowTok{for}\NormalTok{(t }\ControlFlowTok{in} \DecValTok{1}\OperatorTok{:}\NormalTok{T) \{}
\NormalTok{  f.mean[t] =}\StringTok{ }\KeywordTok{mean}\NormalTok{(result.f[t,])}
\NormalTok{  f.var[t] =}\StringTok{ }\KeywordTok{var}\NormalTok{(result.f[t,])}
\NormalTok{\}}
\CommentTok{# Calculate 95% confidence bounds}
\ControlFlowTok{for}\NormalTok{(t }\ControlFlowTok{in} \DecValTok{1}\OperatorTok{:}\NormalTok{T) \{}
\NormalTok{  z =}\StringTok{ }\KeywordTok{qnorm}\NormalTok{(}\FloatTok{0.025}\NormalTok{)}
\NormalTok{  conf.upper[t] =}\StringTok{ }\NormalTok{f.mean[t] }\OperatorTok{+}\StringTok{ }\NormalTok{z }\OperatorTok{*}\StringTok{ }\KeywordTok{sqrt}\NormalTok{(f.var[t]) }
\NormalTok{  conf.lower[t] =}\StringTok{ }\NormalTok{f.mean[t] }\OperatorTok{-}\StringTok{ }\NormalTok{z }\OperatorTok{*}\StringTok{ }\KeywordTok{sqrt}\NormalTok{(f.var[t]) }
\NormalTok{\}}
\CommentTok{# Plotting}
\NormalTok{t =}\StringTok{ }\KeywordTok{seq}\NormalTok{(}\DataTypeTok{from =} \DecValTok{1}\NormalTok{, }\DataTypeTok{to =}\NormalTok{ T, }\DataTypeTok{by =} \DecValTok{1}\NormalTok{)}
\KeywordTok{plot}\NormalTok{(t, f.mean, }\DataTypeTok{type =} \StringTok{"l"}\NormalTok{, }\DataTypeTok{ylab =} \StringTok{"y"}\NormalTok{)}
\KeywordTok{points}\NormalTok{(t, y)                      }
\KeywordTok{lines}\NormalTok{(t, conf.lower, }\DataTypeTok{col =} \StringTok{"red"}\NormalTok{)}
\KeywordTok{lines}\NormalTok{(t, conf.upper, }\DataTypeTok{col =} \StringTok{"red"}\NormalTok{)}
\end{Highlighting}
\end{Shaded}

\includegraphics{Oblig2-problemB_files/figure-latex/unnamed-chunk-2-2.pdf}

\begin{Shaded}
\begin{Highlighting}[]
\CommentTok{# Estimate of pi(eta_10|y)}
\NormalTok{f_}\DecValTok{10}\NormalTok{ =}\StringTok{ }\NormalTok{result.f[}\DecValTok{10}\NormalTok{,]}
\KeywordTok{truehist}\NormalTok{(f_}\DecValTok{10}\NormalTok{)}
\end{Highlighting}
\end{Shaded}

\includegraphics{Oblig2-problemB_files/figure-latex/unnamed-chunk-2-3.pdf}
The histogram shows an estimate for \(\pi(\theta\mid\mathbf{y})\). In
the plot the data points are plotted as circles. The black line is
plotted using the estimates of the smooth effects. The red lines are the
\(95\%\) confidence bounds. Almost all the data points are within the
bounds.

\hypertarget{section-2}{%
\subsection{3.}\label{section-2}}

\begin{Shaded}
\begin{Highlighting}[]
\NormalTok{pi_theta_y =}\StringTok{ }\ControlFlowTok{function}\NormalTok{(theta.grid, y) \{}
\NormalTok{  pi =}\StringTok{ }\KeywordTok{rep}\NormalTok{(}\DecValTok{0}\NormalTok{,}\KeywordTok{length}\NormalTok{(theta.grid))}
\NormalTok{  T =}\StringTok{ }\DecValTok{20}
  \ControlFlowTok{for}\NormalTok{(i }\ControlFlowTok{in} \DecValTok{1}\OperatorTok{:}\KeywordTok{length}\NormalTok{(pi))\{}
\NormalTok{    theta =}\StringTok{ }\NormalTok{theta.grid[i]}
\NormalTok{    Q =}\StringTok{ }\KeywordTok{make.Q}\NormalTok{(T, theta)}
\NormalTok{    deter =}\StringTok{ }\KeywordTok{det}\NormalTok{(}\KeywordTok{solve}\NormalTok{(Q}\OperatorTok{+}\KeywordTok{diag}\NormalTok{(T)))}
\NormalTok{    pi[i] =}\StringTok{ }\NormalTok{theta}\OperatorTok{^}\NormalTok{(T}\OperatorTok{/}\DecValTok{2-1}\NormalTok{) }\OperatorTok{*}\StringTok{ }\KeywordTok{exp}\NormalTok{(}\OperatorTok{-}\NormalTok{theta) }\OperatorTok{*}\StringTok{ }\NormalTok{deter}\OperatorTok{^}\NormalTok{(}\FloatTok{0.5}\NormalTok{) }\OperatorTok{*}\StringTok{ }\KeywordTok{exp}\NormalTok{(}\OperatorTok{-}\FloatTok{0.5} \OperatorTok{*}\StringTok{ }\KeywordTok{t}\NormalTok{(y) }\OperatorTok\StringTok{ }\NormalTok{(}\KeywordTok{diag}\NormalTok{(T)}\OperatorTok{-}\KeywordTok{solve}\NormalTok{(Q}\OperatorTok{+}\KeywordTok{diag}\NormalTok{(T))) }\OperatorTok\StringTok{ }\NormalTok{y)}
\NormalTok{  \}}
  \KeywordTok{return}\NormalTok{(pi)}
\NormalTok{\}}

\NormalTok{thetas =}\StringTok{ }\KeywordTok{seq}\NormalTok{(}\DataTypeTok{from =} \DecValTok{0}\NormalTok{, }\DataTypeTok{to =} \DecValTok{9}\NormalTok{, }\DataTypeTok{by =} \FloatTok{0.1}\NormalTok{)}
\NormalTok{pi =}\StringTok{ }\KeywordTok{pi_theta_y}\NormalTok{(thetas, y)}
\KeywordTok{plot}\NormalTok{(thetas, pi)}
\end{Highlighting}
\end{Shaded}

\includegraphics{Oblig2-problemB_files/figure-latex/unnamed-chunk-3-1.pdf}

\begin{Shaded}
\begin{Highlighting}[]
\NormalTok{mode =}\StringTok{ }\KeywordTok{optimise}\NormalTok{(pi_theta_y,}\DataTypeTok{y=}\NormalTok{y,}\DataTypeTok{lower =} \DecValTok{0}\NormalTok{, }\DataTypeTok{upper =} \DecValTok{9}\NormalTok{,}\DataTypeTok{maximum =} \OtherTok{TRUE}\NormalTok{)}\OperatorTok{$}\NormalTok{maximum}
\NormalTok{mode}
\end{Highlighting}
\end{Shaded}

\begin{verbatim}
## [1] 1.25643
\end{verbatim}

\hypertarget{section-3}{%
\subsection{4.}\label{section-3}}

\begin{Shaded}
\begin{Highlighting}[]
\KeywordTok{set.seed}\NormalTok{(}\DecValTok{0}\NormalTok{)}
\NormalTok{pi_etai_y_theta =}\StringTok{ }\ControlFlowTok{function}\NormalTok{(etai.vec, theta, y) \{}
\NormalTok{  i =}\StringTok{ }\DecValTok{10}
\NormalTok{  T =}\StringTok{ }\KeywordTok{length}\NormalTok{(y)}
\NormalTok{  Q =}\StringTok{ }\KeywordTok{make.Q}\NormalTok{(T, theta)}
\NormalTok{  A =}\StringTok{ }\NormalTok{Q }\OperatorTok{+}\StringTok{ }\KeywordTok{diag}\NormalTok{(T)}
\NormalTok{  a_ii =}\StringTok{ }\NormalTok{A[i,i]}
\NormalTok{  mean =}\StringTok{ }\NormalTok{y[i] }\OperatorTok{/}\StringTok{ }\NormalTok{a_ii}
\NormalTok{  var =}\StringTok{ }\DecValTok{1} \OperatorTok{/}\StringTok{ }\NormalTok{a_ii}
\NormalTok{  pi =}\StringTok{ }\KeywordTok{dnorm}\NormalTok{(etai.vec, }\DataTypeTok{mean =}\NormalTok{ mean, }\DataTypeTok{sd=} \KeywordTok{sqrt}\NormalTok{(var))}
  \KeywordTok{return}\NormalTok{(pi)}
\NormalTok{\}}


\NormalTok{pi_etai_y =}\StringTok{ }\ControlFlowTok{function}\NormalTok{(y,theta.grid, eta.grid) \{}
\NormalTok{  sums =}\StringTok{ }\KeywordTok{rep}\NormalTok{(}\DecValTok{0}\NormalTok{, }\KeywordTok{length}\NormalTok{(eta.grid))}
\NormalTok{  step =}\StringTok{ }\NormalTok{theta.grid[}\DecValTok{2}\NormalTok{]}\OperatorTok{-}\NormalTok{theta.grid[}\DecValTok{1}\NormalTok{]}
\NormalTok{  theta_y =}\StringTok{ }\KeywordTok{pi_theta_y}\NormalTok{(theta.grid, y)}
  \ControlFlowTok{for}\NormalTok{(j }\ControlFlowTok{in}\NormalTok{ (}\DecValTok{1}\OperatorTok{:}\KeywordTok{length}\NormalTok{(theta.grid))) \{}
\NormalTok{    theta =}\StringTok{ }\NormalTok{theta.grid[j]}
\NormalTok{    sums =}\StringTok{ }\NormalTok{sums }\OperatorTok{+}\StringTok{ }\KeywordTok{pi_etai_y_theta}\NormalTok{(eta.grid, theta, y) }\OperatorTok{*}\StringTok{ }\NormalTok{theta_y[j] }\OperatorTok{*}\StringTok{ }\NormalTok{step}
\NormalTok{  \}}
  \KeywordTok{return}\NormalTok{(sums)}
\NormalTok{\}}


\NormalTok{thetas =}\StringTok{ }\KeywordTok{seq}\NormalTok{(}\DataTypeTok{from =} \DecValTok{0}\NormalTok{, }\DataTypeTok{to =} \DecValTok{9}\NormalTok{, }\DataTypeTok{by =} \FloatTok{0.1}\NormalTok{)}
\NormalTok{eta.grid =}\StringTok{ }\KeywordTok{seq}\NormalTok{(}\OperatorTok{-}\DecValTok{1}\NormalTok{,}\DecValTok{1}\NormalTok{,}\FloatTok{0.01}\NormalTok{)}
\NormalTok{etai_y =}\StringTok{ }\KeywordTok{pi_etai_y}\NormalTok{(y,thetas,eta.grid)}

\KeywordTok{plot}\NormalTok{(eta.grid,etai_y)}
\end{Highlighting}
\end{Shaded}

\includegraphics{Oblig2-problemB_files/figure-latex/unnamed-chunk-4-1.pdf}

\hypertarget{section-4}{%
\subsection{5.}\label{section-4}}

\begin{Shaded}
\begin{Highlighting}[]
\KeywordTok{library}\NormalTok{(INLA)}
\end{Highlighting}
\end{Shaded}

\begin{verbatim}
## Loading required package: sp
\end{verbatim}

\begin{verbatim}
## Warning: package 'sp' was built under R version 3.6.3
\end{verbatim}

\begin{verbatim}
## Loading required package: parallel
\end{verbatim}

\begin{verbatim}
## This is INLA_19.09.03 built 2019-09-03 09:03:02 UTC.
## See www.r-inla.org/contact-us for how to get help.
\end{verbatim}

\begin{Shaded}
\begin{Highlighting}[]
\NormalTok{T =}\StringTok{ }\DecValTok{20}
\NormalTok{t =}\StringTok{ }\KeywordTok{seq}\NormalTok{(}\DataTypeTok{from =} \DecValTok{1}\NormalTok{, }\DataTypeTok{to =}\NormalTok{ T, }\DataTypeTok{by =} \DecValTok{1}\NormalTok{)}
\NormalTok{data =}\StringTok{ }\KeywordTok{data.frame}\NormalTok{(}\DataTypeTok{y =}\NormalTok{ y, }\DataTypeTok{t =}\NormalTok{ t)}

\NormalTok{thetahyper =}\StringTok{ }\KeywordTok{list}\NormalTok{(}\DataTypeTok{theta =} \KeywordTok{list}\NormalTok{(}\DataTypeTok{prior =} \StringTok{"log.gamma"}\NormalTok{, }\DataTypeTok{param =} \KeywordTok{c}\NormalTok{(}\DecValTok{1}\NormalTok{, }\DecValTok{1}\NormalTok{)))}
\NormalTok{formula =}\StringTok{ }\NormalTok{y ∼ }\KeywordTok{f}\NormalTok{(t, }\DataTypeTok{model =} \StringTok{"rw2"}\NormalTok{, }\DataTypeTok{hyper =}\NormalTok{ thetahyper, }\DataTypeTok{constr =} \OtherTok{FALSE}\NormalTok{) }\OperatorTok{-}\StringTok{ }\DecValTok{1}
\NormalTok{result1 =}\StringTok{ }\NormalTok{INLA}\OperatorTok{::}\KeywordTok{inla}\NormalTok{(}\DataTypeTok{formula =}\NormalTok{ formula, }\DataTypeTok{family =} \StringTok{"gaussian"}\NormalTok{, }\DataTypeTok{data =}\NormalTok{ data, }\DataTypeTok{control.family =} \KeywordTok{list}\NormalTok{(}\DataTypeTok{hyper=}\KeywordTok{list}\NormalTok{(}\DataTypeTok{prec =}\KeywordTok{list}\NormalTok{(}\DataTypeTok{initial=}\DecValTok{0}\NormalTok{,}\DataTypeTok{fixed=}\OtherTok{TRUE}\NormalTok{))))}

\KeywordTok{plot}\NormalTok{(result1}\OperatorTok{$}\NormalTok{summary.random}\OperatorTok{$}\NormalTok{t}\OperatorTok{$}\NormalTok{mean)}
\end{Highlighting}
\end{Shaded}

\includegraphics{Oblig2-problemB_files/figure-latex/unnamed-chunk-5-1.pdf}

\begin{Shaded}
\begin{Highlighting}[]
\KeywordTok{plot}\NormalTok{(result1}\OperatorTok{$}\NormalTok{marginals.hyperpar}\OperatorTok{$}\StringTok{`}\DataTypeTok{Precision for t}\StringTok{`}\NormalTok{, }\DataTypeTok{xlim =}\KeywordTok{c}\NormalTok{(}\DecValTok{0}\NormalTok{,}\DecValTok{9}\NormalTok{))}
\end{Highlighting}
\end{Shaded}

\includegraphics{Oblig2-problemB_files/figure-latex/unnamed-chunk-5-2.pdf}

\begin{Shaded}
\begin{Highlighting}[]
\KeywordTok{summary}\NormalTok{(result1)}
\end{Highlighting}
\end{Shaded}

\begin{verbatim}
## 
## Call:
##    c("INLA::inla(formula = formula, family = \"gaussian\", data = data, ", 
##    " control.family = list(hyper = list(prec = list(initial = 0, ", " 
##    fixed = TRUE))))") 
## Time used:
##     Pre = 0.475, Running = 0.178, Post = 0.0632, Total = 0.716 
## Random effects:
##   Name     Model
##     t RW2 model
## 
## Model hyperparameters:
##                 mean    sd 0.025quant 0.5quant 0.975quant mode
## Precision for t 1.79 0.941      0.509     1.61       4.12 1.26
## 
## Expected number of effective parameters(stdev): 7.84(1.07)
## Number of equivalent replicates : 2.55 
## 
## Marginal log-Likelihood:  -35.89
\end{verbatim}

\begin{Shaded}
\begin{Highlighting}[]
\KeywordTok{plot}\NormalTok{(result1}\OperatorTok{$}\NormalTok{marginals.random}\OperatorTok{$}\NormalTok{t}\OperatorTok{$}\NormalTok{index}\FloatTok{.10}\NormalTok{,}\DataTypeTok{xlim=}\KeywordTok{c}\NormalTok{(}\OperatorTok{-}\DecValTok{4}\NormalTok{,}\DecValTok{4}\NormalTok{))}
\end{Highlighting}
\end{Shaded}

\includegraphics{Oblig2-problemB_files/figure-latex/unnamed-chunk-5-3.pdf}

\end{document}
